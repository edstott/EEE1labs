\documentclass{standalone}
\usepackage[utf8]{inputenc}
\usepackage{amsmath}
\usepackage[europeanresistors, europeanports]{circuitikz}
\usetikzlibrary{calc,positioning}

\begin{document}

\begin{circuitikz}[scale=1]
    \ctikzset{multipoles/thickness=3}
    \ctikzset{multipoles/dipchip/width=1}
    \def \ctikz@hook@start@draw@default {}
    
    \node [dipchip,num pins=6, hide numbers, no topmark,external pins width=0](state){REG2};
    \node [right] at (state.bpin 1) {D};
    \node [left] at (state.bpin 6) {Q};
    \draw (state.bpin 3) ++(0,0.2) -- ++(0.2,-0.2) -- ++(-0.2,-0.2);
    
    \ctikzset{multipoles/dipchip/width=1.5}
    \node [dipchip,num pins=6, hide numbers, no topmark,external pins width=0,left=2 of state](rom){ROM64x6};
    \node [right] at (rom.bpin 1) {A};
    \node [left] at (rom.bpin 6) {D};

    \draw (rom.bpin 6) -- ++(0.5,0) coordinate (dfork)
    -- (state.bpin 1) node[midway,above] {5:4}
    (dfork) |- ++(1,-2) node[above] {3:0}
    to[short,-o] ++(1,0) node[right] {OUT};

    \draw (rom.bpin 1) -- ++(-0.5,0) coordinate (afork)
    -- ++(0,0.5) -- ++(-1,0) node[midway,above] {5:4}
    coordinate (csbend)
    (state.bpin 6) -| ++(1,1) -| (csbend)
    (afork) -- ++(0,-0.5) -- ++(-1,0) node[midway,above] {3:0}
    to[short,-o] ++(-1,0) node[left]{PB[4:1]};
    
\end{circuitikz}

\end{document}
