\documentclass{standalone}
\usepackage[utf8]{inputenc}
\usepackage{amsmath}
\usepackage[europeanresistors, europeanports]{circuitikz}
\usepackage{adjustbox}
\usetikzlibrary{calc,positioning}

\begin{document}
\trimbox{-.25cm -.25cm -.25cm -.25cm}{
    \begin{circuitikz}[scale=1]
        \ctikzset{multipoles/thickness=3}
        \ctikzset{multipoles/dipchip/width=1}
        \draw (0,0) node[dipchip,num pins=6, hide numbers, no topmark,external pins width=0](headx){HEADX};
        \node [left] at (headx.bpin 6) {POS};
        \draw (headx.bpin 3) ++(0,0.2) -- ++(0.2,-0.2) -- ++(-0.2,-0.2);
        
        \draw (0,-2) node[dipchip,num pins=6, hide numbers, no topmark,external pins width=0](heady){HEADY};
        \node [left] at (heady.bpin 6) {POS};
        \draw (heady.bpin 3) ++(0,0.2) -- ++(0.2,-0.2) -- ++(-0.2,-0.2);

        \coordinate [right=1] (mw) at ($ (headx.bpin 6) !0.5! (heady.bpin 6) $);
        
        \node [dipchip,num pins=6, hide numbers, no topmark,external pins width=0,right=of mw,anchor=bpin 1](tail){REG};
        \node [right] at (tail.bpin 1) {D};
        \node [left] at (tail.bpin 6) {Q};
        \draw (tail.bpin 3) ++(0,0.2) -- ++(0.2,-0.2) -- ++(-0.2,-0.2);

        \draw[very thick] %add/sub input B mux
            coordinate [right = of tail.bpin 6] (m3B)
            coordinate [above = of m3B] (m3A)
            coordinate [right=1.2] (m3Y) at ($ (m3A) !0.5! (m3B) $)
            coordinate [above = of m3A] (m3nw)
            (m3nw) -- ++(1.2,-0.5) node[midway,below=1.5mm](mux3){\textbf{MUX2}}
            -- ++(0,-2)
            -- ++(-1.2,-0.5) node[midway](m3S){}
            -- (m3nw)
            ;
        \node[right=-1mm of m3A]{A};
        \node[right=-1mm of m3B]{B};
        \node[left=-1mm of m3Y]{Y};
        \node[above=0 of m3S]{S};

    
        \draw (headx.bpin 6) -| ($ (mw) + (-0.3,0.3) $) node [above right] {3..0} -| (mw);
        \draw (heady.bpin 6) -| ($ (mw) + (-0.3,-0.3) $) node [below right] {7..4} -| (mw);
        \draw (mw) -- (tail.bpin 1);
        
    \end{circuitikz}
}

\end{document}
