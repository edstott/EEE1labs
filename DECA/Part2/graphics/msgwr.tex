\documentclass{standalone}
\usepackage[utf8]{inputenc}
\usepackage{amsmath}
\usepackage[europeanresistors, europeanports]{circuitikz}
\usetikzlibrary{calc,positioning}

\begin{document}

\begin{circuitikz}
	\footnotesize
	\ctikzset{multipoles/dipchip/width=1.5}
	\draw
		(0,0)
		node[dipchip, num pins=6, hide numbers, no topmark, external pins width=0,anchor=pin 6](tmr){\bfseries CTR}
		node[right] at (tmr.bpin 1) {D}
% 		node[above] at (tmr.s) {RSTn}
		node[left] at (tmr.bpin 6) {Q};
	\draw (tmr.bpin 3) ++(0,0.2) -- ++(0.2,-0.2) -- ++(-0.2,-0.2);
	\draw
		node[dipchip, num pins=6, hide numbers, no topmark, external pins width=0,anchor=pin 1,right=of tmr](rom){\bfseries ROM}
		node[right] at (rom.bpin 1) {ADDR}
		node[left] at (rom.bpin 6) {DOUT};
	\draw (rom.bpin 3) ++(0,0.2) -- ++(0.2,-0.2) -- ++(-0.2,-0.2);

	\draw[ultra thick] (tmr.pin 6) -- (rom.pin 1) coordinate[midway](m);
	\draw
		($(m) + (-0.1,-0.1)$) -- ($(m) + (0.1,0.1)$)
		node[above=0 of m]{4};
	
	\draw[ultra thick] (tmr.pin 1) -- ++(-1,0) coordinate(mlab) node[left]{\texttt{<MSG\_LEN>}};

	\coordinate[below=2 of mlab] (rlab);
% 	\draw (tmr.s) -- (tmr.s|-rlab) -- (rlab) node[left] {RSTn};

	\draw[ultra thick] (rom.pin 6) -- ++(1,0) coordinate[midway](m) node[right]{OUT};
	\draw
		($(m) + (-0.1,-0.1)$) -- ($(m) + (0.1,0.1)$)
		node[above=0 of m]{7};
	

\end{circuitikz}

\end{document}
