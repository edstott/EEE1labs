\documentclass{standalone}
\usepackage[utf8]{inputenc}
\usepackage{amsmath}
\usepackage[europeanresistors, europeanports]{circuitikz}
\usetikzlibrary{calc,positioning}

\begin{document}

\begin{circuitikz}[american voltages]

	\small

	\draw
		node[oscillator,box] (sig) {}
	;

	\node[below left=0 of sig.south] {GND};
	\node[above left=0 of sig.north] {VCC};
	\node[above right=0 of sig.east] {OUT};
	\node[below right=0 of sig.east] {sig. 0};

	\draw (sig.east) -- ++(2,0)coordinate[pos=0.6](br1)  coordinate[pos=0.8](br)
		to[twoport,n=phone] ++(2,0)
		to ++(0,-1) node[ground] {}
		;
		
	\draw (br) to[short,-o] ++(0,5mm) node[above] {$V_\text{in}$};

	\draw (sig.south) to ++(0,-2em) node[ground] {};
	\draw (sig.east) to (phone.west);
	\draw (sig.north) to[short,-o] ++(0,2em) node[above] {5V};

	\draw (phone.center) ++(-4mm,-2.5mm) rectangle ++(1.5mm,5mm)
	-- ++(1.5mm,1mm) -- ++(0,-7mm) -- ++(-1.5mm,1mm)
	(phone.center) ++(4mm,-2.5mm) rectangle ++(-1.5mm,5mm)
	-- ++(-1.5mm,1mm) -- ++(0,-7mm) -- ++(1.5mm,1mm)
	(phone.left) node[below left]{L}
	(phone.right) node[below right]{R}
	;



\end{circuitikz}

\end{document}